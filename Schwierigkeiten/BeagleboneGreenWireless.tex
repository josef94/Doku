\subsection{Beaglebone Green Wireless}
Die Erfahrung mit dem Umgang von Entwicklungsboards war am Anfang der Bachelorarbeit praktisch nicht vorhanden. Dies spiegelte sich auch beim Umgang mit dem Beaglebone Green Wireless wieder. Es handelte sich dabei um ein Entwicklungsboard, welches ohne Bildschirmausgabe verwendet werden musste. Es bestand lediglich die Möglichkeit, die Kommandozeile zu verwenden und dort Eingaben zu machen. Da sich "'Fast and Curious"' um die Analyse von Fahrzeugen mithilfe von Kameras beschäftigte, war es schwierig, eine Bildverarbeitung ohne visuelle Ausgaben durchzuführen. Damit dies dennoch möglich war, wurde ein HDMI Cape bestellt, um die Kameraausgabe auf dem Bildschirm betrachten zu können. Leider funktioniert das HDMI Cape nur für das Beaglebone Green, nicht aber für das Beaglebone Green Wireless. Aus diesem Grund konnte auch damit keine visuelle Ausgabe erreicht werden. 
Alternativ wurde auch das Beaglebone Black getestet, da es mit einem HDMI-Ausgang ausgestattet ist. Der Unterschied zwischen diesen beiden Boards war jedoch so gross, dass die Zeit lieber für das eigentliche Board, das Beaglebone Green Wireless, verwendet wurde.\\\\ 
Bis die erste Software auf dem Beaglebone Green Wireless lief, verging einige Zeit. Die erste funktionsfähige Software zum Ansteuern der Kamera wurde mithilfe von Visual Studio 2015 auf dem Computer unter Windows programmiert und getestet. Anschliessend wurde es auf einer virtuellen Maschine unter Eclipse und Linux optimiert und erneut getestet. Nach einigen weiteren Optimierungen konnte es auf dem Beaglebone Green Wireless zum Laufen gebracht werden. Die erste funktionsfähige Software nahm mithilfe der USB Kamera und OpenCV ein Bild auf und legte es im internen Speicher ab. Das Ergebnis konnte danach auf die externe Speicherkarte verschoben und auf dem Laptop zum ersten Mal betrachtet werden.