\subsection{Beaglebone Green Wireless}
Die Erfahrung mit dem Umgang von Entwicklungsboards war zum Anfang der Bachelorarbeit praktisch nicht vorhanden. Dies wiederspiegelte sich auch beim Umgang mit dem Beaglebone Green Wireless. Dabei handelte es sich um ein Entwicklungsboard, welches ohne Bildschirmausgabe verwendet werden musste. Es bestand lediglich die Möglichkeit, die Kommandozeile zu verwenden und dort Eingaben zu machen. Da sich Fast and Curious um die Analyse von Fahrzeugen mithilfe von Kameras beschäftigte, war es ebenso schwierig, eine Bildverarbeitung ohne visuelle Ausgaben durchzuführen, um zu sehen, was genau passierte. Damit dies dennoch möglich war, wurde ein HDMI Cape bestellt, um die Kameraausgabe auf dem Bildschirm betrachten zu können. Leider handelte es sich beim HDMI Cape um ein Cape für das Beaglebone Green, nicht aber für das Bealgebone Green Wireless. Aus diesem Grund konnte ebenfalls keine Visuelle Ausgabe erreicht werden. Da das Beaglebone Black mit einem HDMI Ausgang ausgestattet war, war dies ebenfalls ein Punkt, welcher probiert wurde. Der Unterschied zwischen diesen beiden Boards war jedoch so verschieden, dass die Zeit lieber für das eigentliche Board, das Beaglebone Green Wireless, verwendet wurde.\\ 
Bis die erste Software auf dem Beaglebone Green Wireless lief, verging einige Zeit. Die erste lauffähige Software zum Ansteuern der Kamera wurde mithilfe von Visual Studio auf dem Computer unter Windows programmiert und getestet, auf einer virtuellen Maschine unter Eclipse und Linux optimiert und erneut getestet und konnte nach einiger weiteren Optimierungen auf dem Beaglebone zum Laufen gebracht werden. Dabei handelte es sich lediglich um ein Bild, welches mithilfe von OpenCV und der USB Kamera geschossen und abgespeichert wurde. Das Ergebnis konnte danach auf die externe Speicherkarte verschoben und auf dem Laptop zum ersten Mal betrachtet werden.