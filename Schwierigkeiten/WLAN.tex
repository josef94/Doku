\subsection{WLAN}
Das WLAN des BeagleBone Green Wireless liess sich nach anfangs leichten Schwierigkeiten sehr rasch mit dem Eduroam Netzwerk der NTB verbinden. Laut den Angaben des Herstellers kann das BeagleBone Green Wireless das integrierte WLAN in einen SoftAP-Mode umwandeln, so dass mit einem externen Gerät auf dieses zugegriffen werden kann. Nach langer Recherche im Internet und sehr vielem herumtüfteln am Board ist es nicht gelungen das integrierte WLAN in diesen Mode zu versetzten. Es blieb keine andere Möglichkeit als die schon vorhandene Website des Herstellers umzufunktionieren und diese selbst zu verwenden. So wurde aus dem Startfenster zum Verbinden mit einem herkömmlichen WLAN, die Startseite des Gerätes. Dies war nur eine Notlösung welche aber sehr gut funktionierte.\\
Nachdem auf den NanoPi NEO umgestiegen wurde, musste das ganze Prozedere von vorne durchgeführt werden. 
Trotz guter Anleitung eines Mitstudenten konnten die WLAN-Adapter am NanoPi nicht in den AP-Mode versetzt werden und somit konnte vom NanoPi kein Hotspot errichtet werden. -Trotz selbem Image, selbem Board und auch selbem WIFI-Adapter funktionierte es nicht-
So musste eine schnelle Lösung gefunden werden. Da jeder ein Smartphone mit Hotspotfunktion besitzt wird nun von diesem ein Hotspot eröffnet und der NanoPi verbindet sich auf diesen. Nachdem dies geschehen ist kann das Gerät verwendet werden. Eine weitere Schwierigkeit war es mit unterschiedlichen WIFI-Adaptern, welche verwendet wurden, das System stabile zum Laufen zu bekommen.
