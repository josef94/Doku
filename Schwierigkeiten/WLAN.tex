\subsection{WLAN}
Das WLAN des Beaglebone Green Wireless liess sich nach leichten Schwierigkeiten sehr rasch mit dem Eduroam Netzwerk der NTB verbinden. Laut den Angaben des Herstellers kann das Beaglebone Green Wireless das integrierte WLAN in einen SoftAP-Mode umwandeln, so dass man mit einem externen Gerät Zugriff darauf erhält. Nach langer Internetrecherche und einigen Testphasen am Board selbst, ist es jedoch nicht gelungen das integrierte WLAN in diesen Mode zu versetzen. Es blieb keine andere Möglichkeit, als die schon vorhandene Webseite des Herstellers umzufunktionieren und diese selbst zu verwenden. So wurde aus dem Startfenster zum Verbinden mit einem herkömmlichen WLAN, die Startseite des Gerätes. Dies war nur eine Notlösung, welche aber sehr gut funktionierte.\\\\
Nach dem Wechsel auf den NanoPi NEO, musste das ganze Prozedere von vorne durchgeführt werden. Trotz einer guten Anleitung eines Mitstudenten konnten auch hier die WLAN-Adapter am NanoPi nicht in den AP-Mode versetzt werden. Somit war es vom NanoPi nicht möglich einen Hotspot zu errichten. Trotz gleichem Image, selbem Board und auch gleichem WIFI-Adapter funktionierte es nicht und es musste schnell eine andere Lösung gefunden werden. Da jeder ein Smartphone mit Hotspotfunktion hatte, wurde von diesen aus ein Hotspot eröffnet, auf welchen sich der NanoPi verbinden konnte. Erst danach war es möglich das Gerät zu verwenden. Eine weitere Hürde war es, mit den unterschiedlichen WIFI-Adaptern zurecht zu kommen. Aufgrund der Verfügbarkeit der WIFI-Adapter wurden schlussendlich drei verschiedene Arten genutzt, wodurch es umso schwieriger war, das System mit allen Adaptern stabil zum Laufen zu bekommen.