\subsection{Radarsensor}
Anfangs wurde die Möglichkeit eines Radarsensors zur Bestimmung der Geschwindigkeit in Betracht gezogen. Jedoch ergab die Auswertung der Daten des Radarsensors einige Schwierigkeiten. Zum einen waren die analogen Eingänge des Beaglebones nur über ihr eigenes BoneScript-Programm erreichbar. Man musste also zuerst über das Programm die Eingänge ansprechen und konnte erst danach die Eingänge über C++ auslesen. Zum anderen sind die Beschreibungen des Herstellers sehr ungenau, da beispielsweise die angegebene Reichweite von ca. zehn Metern nie erreicht wurde. Es wurde lediglich eine Reichweite von knapp sechs Metern erreicht. Dies stellte ein Problem dar, da die Geräte auf einer Höhe von sechs Metern an den Strassenlaternen befestigt werden. Zwangsläufig muss die Reichweite des Sensors dann erheblich grösser sein. Ein weiterer Nachteil des Beaglebones mit dem verwendeten Radarsensor waren die unterschiedlichen Spannungen. Der Radarsensor lieferte eine Ausgangsspannung von 0-5 V, wobei das Beaglebone nur analoge Eingänge mit 3.3 V besitzt. Deshalb musste zusätzlich ein Spannungsteiler eingelötet werden. \\\\
Mit der Verwendung des NanoPi NEO fiel schlussendlich der gesamte Radarsensor aus dem Konzept, weil dieser keine intern analogen Eingänge besitzt. Es musste also eine neue Methode gefunden werden, um die Geschwindigkeit der Verkehrsteilnehmer zu bestimmen.
