\subsection{Radarsensor}
Am Anfang wurde die Möglichkeit eines Radarsensors zur Bestimmung der Geschwindigkeit in Betracht gezogen. Jedoch ergab die Auswertung der Daten des Radarsensors einige Schwierigkeiten. Zum einen waren die analogen Eingänge des BeagleBones nur über ihr eigenes BoneScript Programm erreichbar, so dass man zuerst über dieses die Eingänge ansprechen musste und dann erst die Eingänge über C++ auslesen konnte. Zum anderen waren die Beschreibungen des Herstellers sehr ungenau, da die Reichweite von ca. zehn Meter nie erreicht wurde. Es wurde eine Reichweite von knapp sechs Meter erreicht, jedoch befinden sich die Geräte schon auf einer Höhe von sechs Meter, so dass die Reichweite des Sensors erheblich grösser sein müsste. Ein weiterer Nachteil des BeagleBones und dem verwendeten Radarsensor waren die unterschiedlichen Spannungen. Der Radarsensor lieferte eine Ausgangsspannung von 0-5V, wobei das BeagleBone nur analoge Eingänge mit 3.3V besitzt, so musste ein Spannungsteiler eingelötet werden. \\
Mit der Verwendung des NanoPi NEO viel der gesamte Radarsensor aus dem Konzept da dieser keine internen analogen Eingänge besitzt und so musste eine neue Methode gefunden werden um die Geschwindigkeit der Verkehrsteilnehmer zu bestimmen.
