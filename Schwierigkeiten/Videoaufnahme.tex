\subsection{Videoaufnahme}
Die Aufnahme der Videos erwies sich ebenfalls als schwierig, wodurch einige Schritte und Möglichkeiten getestet wurden, bis die Lösung mit "'Avconv"' gefunden wurde. Die erste Idee war die direkte Verarbeitung des aufgenommenen Videomaterials. Dafür wurde mithilfe von OpenCV ein Frame nach dem anderen aufgenommen und verarbeitet. Aufgrund der Prozessorgeschwindigkeit des Beaglebones wurde diese Idee jedoch schnell wieder verworfen, da mit dieser Methode nur etwa ein Bild pro Sekunde erreicht wurde. Die nächste Möglichkeit bestand darin, ebenfalls mit OpenCV die Videos aufzunehmen und ohne Verarbeitung direkt zu speichern. Somit hätte man die Verarbeitung extern auf dem Computer durchführen müssen. Jedoch wurden selbst mit diese Variante nur etwa zehn Bilder pro Sekunde erzielt, was für den eigentlichen Verwendungszweck immer noch viel zu wenig war. Da es mit OpenCV nicht möglich ist, über zehn FPS zu kommen, wurden andere Programme unter Linux gesucht, welche für eine Videoaufnahme verwendet werden konnten. Daraufhin wurde das Programm "'Streamer"' getestet. Leider konnten auch mit diesem Programm die gewünschten Resultate nicht erreicht werden. Sobald eine Bildrate von mehr als 15 angegeben wurde, gingen einzelne Bilder verloren. Aus diesem Grund war das Video schlussendlich nicht zu gebrauchen. Das Programm zeigte aber, dass eine Videoaufnahme ohne OpenCV vielversprechend sein könnte. Deshalb wurden anschliessend noch weitere Programme ausgetestet. \cite{Streamer}\\\\
Im Anschluss wurde das Programm "'Mjpg-Streamer"' genauer analysiert. Für eine Videoaufnahme mit anschliessender Speicherung war das Programm nicht gut genug, da auch hier einzelne Bilder verloren gingen. Entwickelt wurde dieses Programm jedoch um zu Streamen. Da es in diesem Bereich sehr viel Potential zeigte, konnte es bei "'Fast and Curious"' zum streamen des Videos auf der Webseite eingesetzt werden. \cite{MjpgStreamer} \\\\
Als ein sehr vielversprechendes Programm zeigte sich "'ffmpeg"'. Damit ist es möglich Einstellungen bis ins kleinste Detail vorzunehmen. Mit diesem Programm wurden etwa 20 FPS erreicht, jedoch auch nicht mehr, egal welche Parameter eingestellt wurden. Ob diese 20 Bilder pro Sekunde ausreichen, wurde in vielen Testaufnahmen mit diesem Programm ausprobiert. Da diese Anzahl an Bildern pro Sekunden eher knapp waren, wurde dennoch nach einem weiteren Programm gesucht. \cite{Ffmpeg} \\\\
Schliesslich wurde das Programm "'Avconv"' gefunden. Dabei handelte es sich um ein ähnliches Programm wie bei "'ffmpeg"', jedoch mit mehr Einstellungen. Es verfügte zudem über eine sehr umfangreiche Dokumentation und hilfreiche Ausgaben während der Aufnahme. Mithilfe dieser Ausgaben konnte das Video besser analysiert und die Aufnahme angepasst werden. Durch diese Anpassungen wurden dann zum ersten Mal Videos mit 25 FPS ohne den Verlust einzelner Bilder erreicht, weshalb die Entscheidung schlussendlich auf dieses Programm fiel. Mit anderen Einstellungen wurde getestet, ob noch höhere Bildraten zu erreichen sind. Jedoch fanden bei 30 Bildern pro Sekunde auch mit "'Avconv"' die ersten Bildverluste statt, weshalb der jetzige Aufnahmeprozess mit einer Bildrate von 25 FPS arbeitet. \cite{Avconv}