\subsection{Videoaufnahme}
Die Aufnahme der Videos erwies sich ebenfalls als schwierig, weshalb einige Schritte und Möglichkeiten probiert wurden, bis die Lösung mit "'Avconv"' gefunden wurde. Die erste Idee war die direkte Verarbeitung des aufgenommenen Videomaterials. Dafür wurde mithilfe von OpenCV Frame um Frame aufgenommen und verarbeitet. Aufgrund der Prozessorgeschwindigkeit des Beaglebones wurde diese Idee jedoch schnell verworfen, da mit dieser Methode nur etwa ein Bild pro Sekunde erreicht wurde. Die nächste Idee war es, ebenfalls mit OpenCV, die Videos aufzunehmen und ohne Verarbeitung direkt zu speichern. Somit hätte die Verarbeitung extern auf dem Computer durchgeführt werden müssen. Das Problem war jedoch, dass selbst mit diese Variante nur etwa zehn Bilder pro Sekunde erzielt wurden, was für den Verwendungszweck immer noch viel zu wenig war. Da es mit OpenCV nicht möglich war, über zehn FPS zu kommen, wurden danach andere Programme unter Linux gesucht, welche für eine Videoaufnahme verwendet werden könnten. Dabei wurde das Programm "'Streamer"' probiert. Leider konnte auch dieses Programm nicht die gewünschten Resultate erzielen. Sobald eine Bildrate von mehr als 15 angegeben wurde, gingen einzelne Bilder verloren. Aus diesem Grund konnte das Video dann nicht verwendet werden. Trotzdem zeigte dieses Programm, dass die Videoaufnahme ohne OpenCV vielversprechend sein könnte, weshalb weitere Programme getestet wurden. \cite{Streamer}\\
Im Anschluss wurde das Programm "'Mjpg-Streamer"' genauer analysiert. Für eine Videoaufnahme mit anschliessendem speichern war das Programm nicht genügend gut, da auch hier einzelne Bilder verloren gingen. Entwickelt wurde dieses Programm jedoch um zu Streamen. Da es in diesem Bereich sehr viel Potential zeigte, konnte es bei Fast and Curious zum Streaming des Videos auf der Webseite eingesetzt werden. \cite{MjpgStreamer} \\
Ein sehr vielversprechendes Programm war "'ffmpeg"'. Dort konnten sehr viele Einstellungen gemacht werden, womit alles bis ins Detail eingestellt werden konnte. Mit diesem Programm wurden etwa 20 FPS erreicht, jedoch nie mehr - egal was für Parameter eingestellt wurden. Viele Testaufnahmen wurden mit diesem Programm durchgeführt, ob diese 20 FPS für den Verwendungszweck ausreichen könnten. Da diese Anzahl an Bildern pro Sekunden eher knapp waren, wurde dennoch nach einem weiteren Programm gesucht. \cite{Ffmpeg} \\
Dabei wurde das Programm "'Avconv"' gefunden. Bei diesem Programm handelte es sich um ein ähnliches Programm wie bei "'ffmpeg"', jedoch konnten noch einige Einstellungen mehr durchgeführt werden. Es verfügte zudem über eine sehr umfangreiche Dokumentation und hilfreiche Ausgaben während der Aufnahme. Mithilfe dieser Ausgaben konnte das Video besser analysiert und die Aufnahme angepasst werden. Durch diese Anpassungen konnten dann zum ersten Mal Videos mit 25 FPS ohne den Verlust einzelner Bilder erreicht werden, weshalb die Entscheidung schlussendlich auf dieses Programm fiel. Ebenfalls wurde probiert, mit anderen Einstellungen noch höhere Bildraten zu erreichen. Jedoch fanden bei 30 FPS auch mit "'Avconv"' die ersten Bildverluste statt, weshalb der jetzige Aufnahmeprozess nun mit einer Bildrate von 25 FPS arbeitet. \cite{Avconv}