\subsection{OpenCV}
Eine weitere Schwierigkeit war es, OpenCV erfolgreich auf dem Beaglebone zu installieren. Mithilfe der offiziellen Paketquellen konnte lediglich die Version 2.4 von OpenCV installiert werden. Da aber die Version 3.2 benötigt wurde, um sämtliche Features zu verwenden, welche für Fast and Curious benutzt wurden, musste die Software anderweitig installiert werden. Es kam soweit, dass ebenfalls geprüft wurde, den Sourcecode von Fast and Curious soweit zu optimieren, dass dieser nur mit der Version 2.4 laufen konnte. Jedoch wurde diese Idee relativ schnell wieder verworfen, da die wichtigsten Methoden, welche von Fast and Curious benutzt wurden, von OpenCV nur unter der Version 3.2 installiert waren. Anfangs war ebenfalls der interne Speicher des Beaglebones mit seinen vier Gigabyte zu knapp, um die riesige Library von OpenCV zu installieren. Aus diesem Grund wurden viele Anlaufe und Versuche benötigt, bis ein lauffähiges OpenCV mit der Version 3.2 auf dem Entwicklungsboard installiert war. OpenCV besteht, wie schon im Kapitel Software erwähnt, aus Hauptmodulen und Extramodulen. Die Hauptmodule konnten nach längerem probieren und herunterlöschen von nicht benötigter Software auf dem Beaglebone installiert werden, jedoch wurden die Extramodule zu diesem Zeitpunkt als nicht notwendig erachtet und aufgrund des Speicherplatzes weggelassen. Im späteren Verlauf der Bachelorarbeit wurde ebenfalls festgestellt, dass einige Zusatzmodule zum Auswerten der Verkehrsteilnehmer benötigt wurden. Aus diesem Grund mussten diese dann ebenfalls zum Laufen gebracht werden.\\
Schlussendlich konnte OpenCV mit der Version 3.2 und den benötigten Zusatzmodulen auf dem Beaglebone installiert werden. Deshalb wurde danach ein Git Repository speziell für OpenCV mit allen notwendigen Modulen und Einstellungen erstellt und seither nur dieses verwendet. Dieses Repository konnte selbst beim Wechsel vom Beaglebone zum NanoPi ohne Probleme weiterverwendet werden. Ebenfalls war der Speicherplatz auf dem NanoPi kein Problem mehr, da dort kein interner Speicher vorhanden ist und die Installation sowieso auf der SD Karte durchgeführt werden musste, welche mehr Speicherplatz hatte.