\subsection{OpenCV}
Eine weitere Hürde war die Bibliothek OpenCV erfolgreich auf dem Beaglebone zu installieren. Mithilfe der offiziellen Paketquellen kann lediglich die Version 2.4 von OpenCV installiert werden. Da jedoch für sämtliche Features, welche für "'Fast and Curious"' benutzt wurden, die Version 3.2 notwendig war, musste die Software anderweitig installiert werden. Es kam soweit, dass ebenfalls geprüft wurde, ob es möglich wäre den Sourcecode von "'Fast and Curious"' soweit zu optimieren, dass er auch mit der Version 2.4 laufen konnte. Jedoch wurde diese Idee relativ schnell wieder verworfen, da die wichtigsten Methoden von OpenCV nur unter der Version 3.2 installiert waren. Zudem war es ein Problem die riesige Library von OpenCV auf dem knappen, internen Speicher des Beaglebones, welcher nur vier Gigabyte gross ist, zu installieren. Aus diesem Grund wurden viele Anläufe und Versuche benötigt, bis ein funktionsfähiges OpenCV, mit der Version 3.2, auf dem Entwicklungsboard installiert war. OpenCV besteht, wie schon im Kapitel Software erwähnt, aus Haupt- und Extramodulen. Die Hauptmodule konnten nach längerem testen und dem Löschen von nicht benötigter Software auf dem Beaglebone installiert werden, jedoch wurden die Extramodule zu diesem Zeitpunkt als nicht notwendig erachtet und aufgrund des Speicherplatzes weggelassen. Im späteren Verlauf der Bachelorarbeit wurde aber festgestellt, dass einige Zusatzmodule zum Auswerten der Verkehrsteilnehmer benötigt wurden. Aufgrund dessen mussten diese dann ebenfalls noch zum Laufen gebracht werden.\\\\
Schlussendlich konnte OpenCV mit der Version 3.2 und den benötigten Zusatzmodulen auf dem Beaglebone installiert werden. Danach wurde ein Git Repository, speziell für OpenCV, mit allen notwendigen Modulen und Einstellungen erstellt und seither nur dieses verwendet. Das Repository konnte selbst beim Wechsel vom Beaglebone zum NanoPi ohne Probleme weiterverwendet werden. Der Speicherplatz war auf dem NanoPi kein Problem mehr, da dort kein interner Speicher vorhanden ist und die Installation auf einer SD Karte mit mehr Speicherkapazität durchgeführt werden konnte.