\subsection{Aufgabenstellung}
Die Aufgabe dieses Projektes war es, ein Gerät zu entwickeln, welches autonom vorbeifahrende Verkehrsteilnehmer erkennt und im Anschluss diverse Operationen auf diese ausführt, um somit eine genaue Identifikation dieser Verkehrsteilnehmer zu erreichen. Das Gerät sollte zwei verschiedene Zähler beinhalten, welche sowohl links als auch rechtsfahrende Fahrzeuge unabhängig voneinander zählen kann. Zudem sollte das Gerät die aufgenommenen Fahrzeuge je nach Grösse in verschiedene Kategorien unterteilen und ebenfalls eine grobe Abschätzung über die momentane Geschwindigkeit aussagen können.  Mit zusätzlichem Zeitstempel, welcher bei der Vorbeifahrt der einzelnen Fahrzeuge aufgenommen wurde, sollten alle gefunden Indikatoren in eine Tabelle eingetragen werden, damit diese für spätere Auswertungen verwendet werden kann.  Ein Gerät sollte etwa eine Woche ohne Unterbruch im Betrieb bleiben und zusätzlich die Speicherkarte in dieser besagten Zeit nicht überfüllen. Das einzelne Gerät soll möglichst einfach aufgestellt werden können und dennoch ein gewisses Monitoring möglich sein. Im besten Fall soll eine Ausrichtung mithilfe des Mobilphones durchgeführt werden können und ebenfalls die momentane Ausrichtung ohne grossen Zeitaufwand kontrolliert und korrigiert werden können.\\
Damit eine Verkehrsverfolgung durchgeführt werden kann, sollen die einzelnen Geräte zu einem System gekoppelt werden können. So sollen die Fahrzeuge an mehreren Orten von mehreren Geräten wiedererkannt werden und damit der stattgefundene Verkehrsfluss der Teilnehmer rekonstruiert werden können. Durch strategisch günstige Positionen der Geräte und geeignete Algorithmen sollen nur ein Bruchteil aller aufgenommenen Fahrzeuge für die Wiederidentifikation eines Teilnehmers in Betracht kommen und somit eine genauere und schnellere und Auswertung erreicht werden. Dabei soll es sich um eine statische Aussage handeln, weshalb nicht zwingend jeder Verkehrsteilnehmer exakt erkannt werden muss. Dies wurde so bestimmt, da die Verkehrsteilnehmer "'verschwinden"' können, falls diese im Ort wohnen oder für längere Zeit anderweitig beschäftigt sind.\\
Damit die Verkehrsverfolgung gelingen kann, müssen neben den bereits erwähnten Indikatoren noch weitere Eigenschaften gefunden werden, um die Fahrzeuge genauer einteilen zu können. Was für Indikatoren der Tabelle hinzugefügt werden, sollte sich im Verlauf dieser Bachelorarbeit zeigen. Eine gewisse Auswertung der aufgenommenen Resultate solle ebenfalls visualisiert werden. Bestenfalls könnte dies mithilfe von Open Source Tools wie beispielsweise Open Street Map geschehen.