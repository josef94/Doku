\subsection{Aufgabenstellung}
Die Aufgabe des Projektes war es, ein Gerät zu entwickeln, welches autonom vorbeifahrende Verkehrsteilnehmer erkennt und im Anschluss diverse Operationen auf diese ausführt, um somit eine genaue Identifikation dieser Verkehrsteilnehmer zu erhalten. Das Gerät sollte zwei verschiedene Zähler beinhalten, welche sowohl links- als auch rechtsfahrende Fahrzeuge unabhängig voneinander zählen kann. Zudem sollte das Gerät die aufgenommenen Fahrzeuge je nach Grösse in verschiedene Kategorien unterteilen und ebenfalls eine grobe Abschätzung über die momentane Geschwindigkeit treffen können. Ein zusätzlicher Zeitstempel, welcher bei der Vorbeifahrt der einzelnen Fahrzeuge aufgenommen wurde, trägt alle gefunden Indikatoren in eine Tabelle ein, damit diese für spätere Auswertungen verwendet werden können. Eine wichtige Voraussetzung war es, dass ein Gerät etwa eine Woche ohne Unterbrechung im Betrieb bleiben kann und zudem die eingesetzte Speicherkarte genügend Kapazität aufweist, damit diese im besagten Zeitraum nicht überfüllt wird. Ausserdem wurde eine möglichst einfache Montage der einzelnen Geräte angestrebt, wobei dennoch ein gewisses Monitoring möglich ist. Im besten Fall soll die Ausrichtung mit einem Mobilphone durchgeführt werden können und somit die momentane Ausrichtung ohne grossen Zeitaufwand kontrolliert und korrigiert werden können.\\\\
Um eine aussagekräftige Verkehrsverfolgung zu erhalten, werden die einzelnen Geräte zu einem kompletten System gekoppelt. So sollen die Fahrzeuge an mehreren Orten von den verschiedenen Geräten wiedererkannt und damit der Bewegungsablauf der einzelnen Verkehrsteilnehmer rekonstruiert werden. Durch strategisch günstige Positionen der Geräte und geeignete Algorithmen soll es möglich sein, dass nur ein Bruchteil aller aufgenommenen Fahrzeuge für die Identifikation eines bestimmten Teilnehmers in Betracht kommen kann. Dadurch wird eine genauere und schnellere Auswertung erreicht. Es muss nicht zwingend jeder Verkehrsteilnehmer erkannt werden, da es sich dabei lediglich um eine statistische Aussage handelt. Diese Entscheidung wurde so festgelegt, da nicht alle Verkehrsteilnehmer zu 100\% verfolgt werden können. Probleme entstehen beispielsweise, wenn diese im Ort wohnen oder für längere Zeit dort verweilen.\\\\
Damit eine Verkehrsverfolgung gelingen kann, müssen neben den bereits erwähnten Indikatoren noch weitere Eigenschaften gefunden werden, um die Fahrzeuge genauer einteilen zu können. Welche zusätzlichen Indikatoren damit gemeint sind, sollte sich im Verlauf dieser Bachelorarbeit zeigen. Letztendlich soll die Auswertung der aufgenommenen Resultate noch visualisiert werden. Bestenfalls könnte dies mithilfe von Open Source Tools wie beispielsweise Open Street Map geschehen.