\subsection{Verwendete Tools und Libraries}
\subsubsection{Mjpg-Streamer}
Beim Mjpg-Streamer handelt es sich um ein Werkzeug für die Kommandozeile, welches es ermöglicht Videos von einer Videoquelle aufzunehmen und diese entweder in einer Datei abzuspeichern oder es auf einem Webserver zu Streamen. Beim Abspeichern eines Videos konnte etwa eine Bildrate von 15 Bilder pro Sekunde erreicht werden, bevor einige Bilder verloren gingen. \\
Grundvoraussetzung für den Mjpg-Streamer ist eine Kamera, welche korrekt eingerichtet wurde. Bei den meisten Modellen muss deshalb der Treiber für "'Video 4 Linux 2"' zuerst installiert werden. Die Installation vom Mjpg-Streamer funktioniert nicht über die offiziellen Paketquellen. Aus diesem Grund müssen zuerst einige andere Pakete von den offiziellen Paketquellen installiert werden, bevor der aktuelle Quelltext der eigentlichen Software heruntergeladen und installiert werden kann. Die Verwendung erfolgt relativ simpel indem zuerst das gewünschte Plugin und danach dessen Parameter übergeben werden kann. Dies wird sowohl für den Eingang aus auch den Ausgang des Videos benötigt. Einige Parameter für den Eingang der Kamera sind neben dem Plugin die Auflösung, der Gerätename, die Bildwiederholrate und die Bildqualität. Bei den Parametern für die Ausgabe handelt es sich um den Speicherort, das Bildintervall oder auch den Port, falls es sich beim Ausgang um einen Webserver handelt. \cite{MjpgStreamer}

\subsubsection{Avconv}
Avconv ist ein sehr schneller Audio und Videokonverter, welcher ebenfalls dazu verwendet werden kann, um Videos mit Bild und Ton aufzunehmen und diese in einer Datei abzuspeichern. Auch dieses Tool kann von der Kommandozeile aus gesteuert werden und liefert viele Möglichkeiten um Konvertierungen von Bild und Audiomaterial durchzuführen. Es kann ebenfalls dazu verwendet werden, Videomaterial in andere Datentypen zu konvertieren oder aber Videomaterial schneller beziehungsweise langsamer laufen zu lassen. Beispielsweise bietet Avconv die Möglichkeit eine der beiden Spuren zu auszuschalten, wenn diese nicht benötigt werden. Es können viele Bearbeitungsmöglichkeiten wie Schneiden eines Videos in eine geqünschte Länge oder das Verbinden von zwei einzelnen Videos zu einem längerem Video durchgeführt werden. Ebenso können mehrere Eingänge und Ausgänge angegeben werden, womit es die Möglichkeit bietet, eine eigene Bildspur und Audiospur zu einem Video zu migrieren und es gleichzeitig in mehrere Videoformate abzuspeichern. Anders als beim Mjpg-Streamer kann Avconv direkt von den offiziellen Paketquellen heruntergeladen und installiert werden. Die Verwendung dieses Tools ist komplizierter als beim Mjpg-Streamer, da es viel mehr Parameter beinhaltet, welche angegeben können. Bei der Videoaufnahme können beispielsweise Parameter wie Auflösung, Eingangsvideogerät, Aufnahmezeit, Bildwiederholungsrate, Bitrate oder Bildqualität angegeben werden. \cite{Avconv}

\subsubsection{OpenCV}
OpenCV ist die Abkürzung für "'Open Source Computer Vision Library"' und ist eine Bibliothek, bei welchem der Code offen zugänglich und nutzbar ist. Diese Software Bibliothek ist für Bild und Videoverarbeitung sowie "'Machine Learning"' und kann in sämtlichen kommerziellen Produkten eingesetzt werden. Die Bibliothek besteht aus mehr als 2500 optimierten Algorithmen und beinhaltet sämtlich gängige Methoden für Bildverarbeitung und "'Machine Learning"'. Zu den Algorithmen gehören beispielsweise das Detektieren und Erkennen von Objekten, Klassifizieren von Menschlichen Aktionen, Aufnehmen von Kamerabewegungen, Erkennen von 3D-Objekten, Produzieren von 3D-Objekten aus einem normalen Kamerabild, Zusammenführen von Bildern um hochauflösende Bilder ganzer Szenen zu erschaffen, um ähnliche Bilder in einem Set von Bildern zu finden, Rote Augen von Bildern zu entfernen, Verfolgen von Augen oder Gesichtern, "'Augmented Reality"' und vieles mehr. OpenCV bietet Schnittstellen für die Programmiersprachen "'C++"', "'C"', "'Python"', "'Java"' und "'MATLAB"' und unterstützt die gängigen Plattformen Windows, Android, Mac Os und Linux. Es wird stetig weiterentwickelt und kann ohne Probleme in einem Projekt eingebunden werden. Die Aktuelle Version ist OpenCV 3.2 und besteht aus Hauptmodulen und Extramodulen. Die normale Installation beinhaltet nur die Hauptmodule, welche jedoch für den normalen Gebrauch ausreichen. Um die Extramodule verwenden zu können müssen diese extra heruntergeladen und in den OpenCV Ordner verschoben werden, bevor die Installation durchgeführt wird. Für die Installation von OpenCV werden einige offizielle Softwarepakete benötigt, welche vorgängig heruntergeladen und installiert werden müssen. Die Installation von OpenCV selber funktioniert nicht über die offiziellen Paketquellen, weshalb die Software manuell heruntergeladen werden muss. Zu diesem Zeitpunkt können auch die gewünschten Zusatzmodule heruntergeladen werden, bevor danach das gesamte OpenCV Paket installiert wird. \cite{OpenCV}

\subsubsection{Apache}
Mit Apache versteht man unter Linux den Apache HTTP Server. Dies ist ein sehr bekannter Web Server, welcher meist mit Skriptsprachen wie PHP oder Datenbanken wie MySQL verwendet werden kann, um Websiten zu hosten und vom Browser aus zugegriffen werden kann. Die Installation des Apache HTTP Servers unter Linux erfolgt lediglich über einen einfachen Befehl und wird dann von den offiziellen Paketquellen heruntergeladen und installiert. Nach der Installation befindet sich der HTTP Server ohne Konfiguration im Autostart und kann nach einem Neustarten der Linux Plattform direkt verwendet werden. Um eigene Internetseiten auf dem Apache zugänglich zu machen, muss lediglich die gewünschte Seite als HTML oder PHP in den richtigen Ordner geschoben werden und kann daraufhin über die IP erreicht werden. \cite{Apache}