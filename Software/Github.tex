\subsection{Github}
Zur Verwaltung der Software wurde Github verwendet. Damit konnten Anpassungen oder Änderungen des Quellcodes einfach von der Kommandozeile, im Webbrowser oder mithilfe eines Programms unter Windows hochgeladen und anschliessend auf anderen Geräten verbreitet werden. Ebenfalls war es jederzeit möglich die Software auf vorherige Versionen des Quellcodes zurückzusetzen, wenn dies notwendig war. Durch "'SourceTree"', einem Programm unter Windows, konnten zudem sämtliche vorgängige Commits rasch und einfach nachgelesen und nachvollzogen werden, da dort zu jedem Commit die spezifischen Änderungen markiert wurden. Ein weiterer Vorteil von Github besteht darin, dass der Quellcode von überall aus erreichbar ist und somit keine verschiedenen Versionen von Codes vorhanden waren. 
Im späteren Verlauf der Bachelorarbeit war es möglich den Quellcode der Geräte, die bereits an den Strassenlaternen angebracht wurden, stetig zu verbessern. Ohne die Geräte zu entfernen und neu ausrichten zu müssen, konnten neue Versionen der Software per WLAN heruntergeladen und installiert werden. Der Quellcode konnte also jederzeit getestet, angepasst und innerhalb weniger Minuten auf den Geräten aktualisiert werden. Die Installationszeit für Updates wurde sehr kurz gehalten, da es mittels Script durchgeführt wurde, welches sämtliche Programme unter Linux überprüft und neue Software nur installiert, falls diese auf dem Board noch nicht vorhanden ist.

\subsubsection{Ordnerstruktur}
Sämtlicher Quellcode der Geräte ist auf dem USB-Stick im Anhang der Arbeit beigelegt. Die nachfolgende Tabelle (\tref{tOrdnerstruktur}) zeigt die Ordnerstruktur des Quellcodes.

\setlength\tabcolsep{5pt}
\renewcommand{\arraystretch}{1.5}
\begin{table}[H]
\centering
\begin{tabular}{|p{0.2 cm} p{4 cm} p{4 cm} p{4 cm}|}
\hline
\multicolumn{4}{|l|}{ \textbf{MakeVideo}} \\ \hline
 &  makeVideo.sh &  & \\ \hline
\multicolumn{4}{|l|}{\textbf{CheckMotion}} \\ \hline
 & CheckMotion.cpp &  &  \\ \hline
 \multicolumn{4}{|l|}{\textbf{VehicleCount}} \\ \hline
 & \begin{tabular}[c]{@{}l@{}}Blob.cpp\\ Blob.h\end{tabular} & \begin{tabular}[c]{@{}l@{}}FileHandler.cpp\\ FileHandler.h\end{tabular} & VehicleCount.cpp \\ \hline
\multicolumn{4}{|l|}{\textbf{Handler}} \\ \hline
 & handler.sh & settings.sh & temperature.sh \\ \hline
\multicolumn{4}{|l|}{\textbf{WLAN}} \\ \hline
 & WLANSettings.sh &  &  \\ \hline
\multicolumn{4}{|l|}{\textbf{Temp}} \\ \hline
 & \begin{tabular}[c]{@{}l@{}}deleteFrames.sh\\ generateCrops.sh\\ generateFeatureVec.sh\end{tabular} & \begin{tabular}[c]{@{}l@{}}index.php\\ rc.local\\ startStream.sh\end{tabular} & \begin{tabular}[c]{@{}l@{}}startVideo.sh\\ temperature.php\end{tabular} \\ \hline
\multicolumn{4}{|l|}{} \\ \hline
 & initialization.sh &  &  \\ \hline
\end{tabular}
\caption{Ordnerstruktur.}
\label{tOrdnerstruktur}
\end{table}

\renewcommand{\arraystretch}{1}
\setlength\tabcolsep{0pt}

\newpage
Das Skript "'makeVideo.sh"' dient dazu, die Videoaufnahme zu starten. Nach Aufruf des Scripts, wird ein Zeitstempel mit aktuellem Datum und Uhrzeit herausgelesen und daraufhin ein 15-Minütiges Video gestartet, welches im Videos-Ordner gespeichert wird.\\\\
Die C++ Datei CheckMotion ist für die Vorverarbeitung zuständig. Nachdem die Datei kompiliert wurde, kann diese ausgeführt werden. Dieses Programm benötigt jedoch den Namen des Videos und den Pfad für die generierten Frames als Argument.\\\\
Sämtliche Dateien im VehicleCount werden benötigt, um die Nachbearbeitung durchzuführen. Auch diese Dateien müssen vorgängig kompiliert werden, damit sie ausführbar sind. Dabei ist VehicleCount die Hauptklasse, Filehandler kümmert sich um das Datei-Handling und Blob um die eigentliche "'Blob-Detection"'. Als Argument benötigt die Hauptklasse den korrekten Pfad des Ordners.\\\\
Die drei Dateien im Ordner "'Handler"' gehören zur Steuerung und Überwachung. Dabei beinhaltet die Datei "'handler.sh"' die Hauptschleife. "'settings.sh"' wird benötigt, um den Live-Stream zu starten, damit das Gerät ausgerichtet werden kann und "'temperature.sh"' liest alle 60 Sekunden die Kerntemperatur aus, welche danach in eine Datei geschrieben wird.\\\\
Die Datei "'WLANSettings.sh"' wird benötigt, um die gewünschten WIFI Einstellungen durchzuführen. Dort kann die SSID und das Passwort des Netzwerks eingetragen werden, auf welches sich das Gerät verbindet.\\\\
Alle Dokumente, die sich im Ordner "'Temp"' befinden, müssen nach herunterladen vom Git Repository noch in den korrekten Ordner verschoben werden. Die PHP Dateien beinhalten die Webseite von "'Fast and Curious"'. Die SH Skripte werden für die Eingaben durch den Benutzer auf der Webseite benötigt. Die letzte Datei, "'rc.local"' ist notwendig, um diverse Programme bereits beim Autostart zu aktivieren, damit kein manuelles anmelden auf dem NanoPi und starten der Skripte nötig ist.\\\\
Damit eine Neuinstallation oder ein Update ohne grosse Probleme durchgeführt werden kann, ist die Datei "'initialization.sh"' vorhanden. Nachdem das Git Repository heruntergeladen wurde, kann dieses Skript ausgeführt werden. Nach dem Starten wird zuerst nach Upgrades und Updates vom sämtlichen Programmen auf dem Board gesucht. Daraufhin werden die notwendigen Ordner erstellt, welche für den Betrieb des Gerätes benötigt werden. Im Anschluss werden OpenCV, Avconv, Apache2 und Mjpg-Streamer installiert, falls diese sich noch nicht auf dem Gerät befinden. Anschliessend werden alle Dateien im Temp Ordner in die richtigen Ordner auf dem Gerät verschoben und dieser danach gelöscht. Die erfolgreiche Löschung des Ordners dient als Indikator, ob alle Dateien korrekt verschoben wurden. Zu guter Letzt werden die C++ Dateien kompiliert und sämtliche Skripte ausführbar gemacht. Erst nachdem all diese Aufgaben abgeschlossen sind, erscheint eine finale Meldung, welche den Benutzer informiert, ob alles erfolgreich abgelaufen ist und das Gerät nach einem Neustart verwendet werden kann.