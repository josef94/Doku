\subsection{Systemstart}
Im Anschluss zur Aufstellung kann das Gerät in Betrieb genommen werden, jedoch muss noch eine genaue Ausrichtung durchgeführt werden, damit die Strasse optimal im Sichtfeld der Kamera ist. Diese Ausrichtung wird erreicht, indem das Mobilphone, wie bei einem Update, als Hotstpot eingeschaltet wird. Sobald die Stromversorgung zum Gerät angeschlossen wurde, wird es versuchen, sich mit dem Hotspot zu verbinden. Nach erfolgreicher Verbindung des Geräts kann die IP ausgelesen werden, welche im Anschluss im Browser eingegeben werden kann, um auf die Website von "'Fast and Curious"' zu kommen. Auf dieser Website kann danach die genaue Ausrichtung des Geräts durchgeführt werden, da dort der Webstream der Kamera zu sehen ist. Das nachfolgende Bild (\fref{bWebsite}) zeigt die Website des Gerätes.

\begin{figure}[H]
  \centering
  \fbox{\includegraphics[width=0.62\textwidth]{Bedienungsanleitung/Website.jpg} }
  \caption{Website von "'Fast and Curious"'}
  \label{bWebsite}
\end{figure} 

Wenn die Feinjustierung des Gerätes abgeschlossen ist, kann die eigentliche Aufnahme der Verkehrsteilnehmer durch den Knopf "'Videoaufnahme starten"' eingeschalten werden. Zu diesem Zeitpunkt wird der Videostream einfrieren, da die Kamera vom Streaming Programm "'Mjpg-Streamer"' an die Videoaufnahme "'Avconv"' übergeben wird.