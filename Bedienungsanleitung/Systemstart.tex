\subsection{Systemstart}
Nach dem Aufstellen kann das Gerät in Betrieb genommen werden, jedoch muss noch eine genauere Ausrichtung durchgeführt werden, damit die Strasse optimal im Sichtfeld der Kamera ist. Diese Ausrichtung wird erreicht, indem das Mobilphone, wie bei einem Update als Hotstpot eingesetzt wird. Sobald die Stromversorgung zum Gerät angeschlossen wurde, wird es versuchen, sich mit dem Hotspot zu verbinden. Nach erfolgreicher Verbindung des Gerätes kann die IP-Adresse, welche im Anschluss im Browser eingegeben wird, ausgelesen werden, um auf die Webseite von "'Fast and Curious"' zu kommen. Auf dieser Webseite kann nun die genaue Ausrichtung des Gerätes erfolgen, da dort der Live-Stream der Kamera zu sehen ist. Das nachfolgende Bild (\fref{bWebsite}) zeigt die Webseite des Gerätes.

\begin{figure}[H]
  \centering
  \fbox{\includegraphics[width=0.62\textwidth]{Bedienungsanleitung/Website.jpg} }
  \caption{Webseite von "'Fast and Curious"'.}
  \label{bWebsite}
\end{figure} 

Wenn die Feinjustierung des Gerätes abgeschlossen ist, kann die eigentliche Aufnahme der Verkehrsteilnehmer durch den Knopf "'Videoaufnahme starten"' gestartet werden. Zu diesem Zeitpunkt wird der Live-Stream einfrieren, da die Kamera vom Streaming Programm "'Mjpg-Streamer"' an die Videoaufnahme "'Avconv"' übergeben wird.