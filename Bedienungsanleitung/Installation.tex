\subsection{Installation}
Um ein neues Gerät zu erstellen, müssen zuerst sämtliche Hardwareteile besorgt und korrekt zusammengebaut werden.\\\\
Beim Zusammenbau der einzelnen Geräte muss lediglich darauf geachtet werden, dass der Spannungswandler polgerecht an der Autobatterie angeschlossen wird. Werden die Drähte vertauscht so kann ein Kurzschluss entstehen und die gesamte Hardware zerstören. Der restliche Zusammenbau ist sehr einfach. Den USB-Hub am NanoPi NEO anschliessen und an diesen dann sowohl Kamera als auch den WIFI-Adapter anbringen. Danach wird lediglich der Strom korrekt angeschlossen. Das Gerät ist danach einsatzbereit.\\\\
\textbf{Achtung: Auf richtige Polung achten! Kurzschlussgefahr!}\\\\
Nachdem das Gerät zusammengebaut wurde, muss im nächsten Schritt die Speicherkarte mit dem Ubuntu Image beschrieben werden. Das verwendete Image der Bachelorarbeit wurde dem USB-Stick dieser Bachelorarbeit beigefügt. Dies kann mit einem dazu geeigneten Programm, beispielsweise "'Win32 Disk Imager"', durchgeführt werden. Im Anschluss muss die beschriebene Karte in den Speicherkartenslot des NanoPi geschoben und dieser dann anschliessend gestartet werden. Die Rechnereinheit sollte zur Installation via Netzwerkkabel verbunden werden um die IP-Adresse beim Router nachschauen zu können. Zudem ist zu diesem Zeitpunkt das WLAN Netzwerk noch nicht konfiguriert. Via "'Putty"' kann über die IP-Adresse auf den NanoPi zugegriffen werden. Für das bereitgestellte Image von Ubuntu lautet der Benutzername "'root"' und das Passwort "'fa"'. Es empfiehlt sich das Standardpasswort zu diesem Zeitpunkt zu ändern, um die Daten auf der Speicherkarte zu schützen. Die benötigten Dateien und sonstige Software können von Github bezogen werden. Sobald die Dateien heruntergeladen wurden muss das Bash-Skript "'initialization.sh"' ausgeführt werden, worauf alles Notwendige installiert wird. Dieser Vorgang dauert etwa drei Stunden, da für OpenCV viel Dateien kompiliert werden müssen.\\\\
Damit das WLAN Netzwerk später funktioniert, muss der Name in "'/etc/network/interfaces"' angepasst werden. Dies wurde nicht im Installationsskript durchgeführt, da für die Bachelorarbeit drei verschiedene WLAN Sticks verwendet wurden. Sobald alles fertig installiert und der Name des Netzwerks angepasst ist, kann das Gerät neu gestartet werden. Somit ist die Installation des Gerätes "'Fast and Curious"' abgeschlossen.