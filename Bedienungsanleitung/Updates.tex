\subsection{Updates}
Updates sind möglichst einfach durchzuführen, da die Software auf einem beliebigen Gerät angepasst und die Änderungen dann anschliessend auf Github hochgeladen werden kann. \\\\
Damit die durchgeführten Änderungen auf die anderen Geräte verbreitet werden können, muss man via "'Putty"' auf das Gerät zugreifen. Dies kann direkt vor Ort an einem Gerät stattfinden, wenn mit dem Mobilphone ein Hotspot errichtet wird. Dazu benötigt der Hotspot jedoch die im Skript "'WLANSettings.sh"' eingegebenen Parameter für SSID und Passwort. Im Anschluss sollte sich das Gerät mit dem Hotspot verbinden und die IP-Adresse angezeigt werden. Wenn der Zugriff auf das Gerät erfolgreich war, muss der Ordner "'BA"' gelöscht und neu von Github heruntergeladen werden. Im Anschluss daran kann das Initialisierungsskript ausgeführt werden, wodurch alles neu installiert, verschoben und kompiliert wird. Nach einem Neustart befindet sich das Gerät auf dem neusten Stand.