\subsection{Updates}
Um Updates möglichst einfach durchzuführen, kann die Software auf einem beliebigen Gerät angepasst und die Änderungen anschliessend auf Github hochgeladen werden. Um diese Änderungen hochladen zu können muss der Benutzername "'josef94"' und das Passwort "'brauentinweg12"' angegeben werden.
Damit die durchgeführten Änderungen auf die anderen Geräte verbreitet werden können, muss via Putty auf das Gerät zugegriffen werden. Dies kann direkt vor Ort an einem Gerät gemacht werden, wenn mit dem Mobilphone ein Hotspot errichtet wird. Dazu benötigt der Hotspot jedoch die im Skript "'WLANSettings.sh"' eingegebenen Parameter für SSID und Passwort. Im Anschluss sollte sich das Gerät mit dem Hotspot verbinden und die IP angezeigt werden. Wenn der Zugriff auf das Gerät erfolgreich war, muss der Ordner "'BA"' gelöscht und neu von Github heruntergeladen werden. Im Anschluss kann das Initialisierungs-Skript ausgeführt werden, wodurch alles neu installiert, verschoben und kompiliert wird. Nach einem Neustart befindet sich das Gerät auf dem neusten Stand.