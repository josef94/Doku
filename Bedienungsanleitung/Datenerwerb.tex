\subsection{Datenerwerb}
Die aufgenommenen Daten können jederzeit, auch während dem Betrieb des Gerätes, heruntergeladen werden. Damit dies ohne grossen Aufwand passieren kann, muss wieder eine Verbindung zum Gerät mithilfe eines Hotspots erstellt werden. Im Anschluss kann der Download Ordner über die Webseite erreicht werden. Damit darin die neusten Daten vorhanden sind, müssen diese vorgängig generiert werden. Dies geschieht, indem die Crops und der Feature Vektor über die zugehörigen Knöpfe auf der Website gezippt und zum Download Ordner übertragen werden. Beim Feature Vektor handelt es sich um eine Microsoft Excel Tabelle, wie sie in im Abschnitt "'Software"' (\tref{tFeatureVektor}) zu sehen ist. Die Crops beinhalten ein Originalbild, ein Differenzbild und ein kleineres Bild mit nur den bewegtem Blob des Verkehrsteilnehmers. Es kann ebenfalls eine Tabelle mit sämtlichen aufgezeichneten Temperaturdaten heruntergeladen werden, damit diese extern ausgewertet werden kann.