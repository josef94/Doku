\subsection{Anforderungen}
An das Gerät und das gesamte System werden diverse Anforderungen gestellt, welche im Folgenden aufgezählt werden.

\subsubsection{Robustheit}
Das System muss eine sehr robuste Auswertung liefern, so dass es möglich ist, bei jeder Wettersituation dieselben Resultate zu erzielen. Im Allgemeinen bedeutet es, dass das System die Verkehrsteilnehmer eindeutig identifizieren können muss, egal ob die Verkehrsteilnehmer voller Schmutz, Regen oder Sonnenstrahlen sind. Das Gerät wird dazu in ein wasserdichtes Gehäuse gehüllt. 

\subsubsection{Laufzeit}
Das Gerät arbeitet autark und ist mit einer externen Spannungsquelle versehen, da nicht an jeder Strassenlaterne ein Anschluss vorhanden ist, um die Geräte dort zu laden. Die Laufzeit der autarken Geräte sollte mehrere Tage betragen, da nur so gewährleistet ist, dass aussagekräftige Ergebnisse im Bezug auf den Verkehrsfluss erziel werden können.  

\subsubsection{Verarbeitungszeit}
Die Verarbeitung der gesammelten Daten sollte beinahe in Echtzeit geschehen, um eventuell die Ergebnisse online darstellen zu können, wenn das System zu einem späteren Zeitpunkt weiter ausgereift wird. Aufgrund dessen wird ein sehr hoher Anspruch an die verwendete Hardware gesetzt. Ebenso muss die Software so angepasst sein, dass auch sie die Daten in Echtzeit verarbeiten kann.

\subsubsection{Kosten}
Da das System aus mehreren Geräten besteht und jede Gemeinde sich das System leisten können soll, muss ein einzelnes Gerät sehr Preisgünstig hergestellt werden. Die Vorgaben an das Gerät lauten, einen Preis von 200 CHF nicht zu überschreiten. Da dann für 1000 CHF fünf solcher Geräte in einem Gebiet installiert werden können. Das System wird dadurch nicht von der Qualität der einzelnen Bauteile definiert, sondern vom Preis und der Anzahl der einzelnen Geräte.
