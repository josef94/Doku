\subsection{Anforderungen}
An das Gerät und das gesamte System werden diverse Anforderungen gestellt welche im Folgenden aufgezählt werden.
\subsubsection{Robustheit}
Das System sollte eine sehr robuste Auswertung liefern, so dass es möglich ist bei jeder Wettersituation dieselben Resultate zu erzielen. Im allgemeinen heisst dies, dass das System die Verkehrsteilnehmer eindeutig identifizieren kann, egal ob die Verkehrsteilnehmer voller Schmutz, Regen oder Sonnenstrahlen sind. Das Gerät wird dazu in ein wasserdichtes Gehäuse gehüllt. 
\subsubsection{Laufzeit}
Das Gerät sollte autark arbeiten und mit einer externen Spannungsquelle versehen werde, da nicht an jeder Strassenlaterne ein Anschluss vorhanden ist um die Geräte direkt zu Speisen. Die Laufzeit der autarken Geräte sollte mehrere Tage betragen, da es sich um ein System handelt welches Daten braucht welche mehr als nur wenige Stunden sind. Um den Verkehrsfluss darstellen zu können muss das Gerät mehrere Tage am Stück einsatzfähig sein.
\subsubsection{Verarbeitungszeit}
Die Verarbeitung der gesammelten Daten sollte beinahe in Echtzeit geschehen, um eventuell bei späteren weiter Entwicklungen online dargestellt werden zu können. Daraus wird ein sehr hoher Anspruch an die eingesetzte Hardware gesetzt. Ebenso muss die Software so angepasst sein, dass diese ebenso in Echtzeit die Daten verarbeitet hat.
\subsubsection{Kosten}
Da das System aus mehreren Geräten besteht und jede Gemeinde sich das System leisten können soll, muss ein einzelnes Gerät ser Preisgünstig hergestellt werden. Die Vorgaben sind das Gerät unter einem Preis von 200CHF herzustellen. Da dann für 1000CHF fünf solcher Geräte in einem Gebiet installiert werden können. Das System wird dadurch nicht von der Qualität der einzelnen Bauteile definiert, sondern von dem Preis und der Anzahl der einzelnen Geräte.
