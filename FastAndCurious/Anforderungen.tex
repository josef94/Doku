\subsection{Anforderungen}
An das Gerät und das gesamte System werden diverse Anforderungen gestellt welche im Folgenden aufgezählt werden.

\subsubsection{Robustheit}
Das System sollte eine sehr robuste Auswertung liefern, so dass es möglich ist, bei jeder Wettersituation dieselben Resultate zu erzielen. Im Allgemeinen heisst dies, dass das System die Verkehrsteilnehmer eindeutig identifizieren kann, egal ob die Verkehrsteilnehmer voller Schmutz, Regen oder Sonnenstrahlen sind. Das Gerät wird dazu in ein wasserdichtes Gehäuse gehüllt. 

\subsubsection{Laufzeit}
Das Gerät sollte autark arbeiten und mit einer externen Spannungsquelle versehen werde, da nicht an jeder Strassenlaterne ein Anschluss vorhanden ist um die Geräte direkt zu Speisen. Die Laufzeit der autarken Geräte sollte mehrere Tage betragen, da es sich um ein System handelt, welches Daten aufzeichnet, die mehr als nur wenige Stunden alt sind. Um den Verkehrsfluss darstellen zu können muss das Gerät mehrere Tage am Stück einsatzfähig sein.

\subsubsection{Verarbeitungszeit}
Die Verarbeitung der gesammelten Daten sollte beinahe in Echtzeit geschehen, um eventuell bei späterer weiterer Entwicklungen die Ergebnisse online darstellen zu können. Daraus wird ein sehr hoher Anspruch an die eingesetzte Hardware gesetzt. Ebenso muss die Software so angepasst sein, dass diese ebenso in Echtzeit die Daten verarbeiten kann.

\subsubsection{Kosten}
Da das System aus mehreren Geräten besteht und jede Gemeinde sich das System leisten können soll, muss ein einzelnes Gerät sehr Preisgünstig hergestellt werden. Die Vorgaben an das Gerät lauten, einen Preis von 200 CHF nicht zu überschreiten. Da dann für 1000 CHF fünf solcher Geräte in einem Gebiet installiert werden können. Das System wird dadurch nicht von der Qualität der einzelnen Bauteile definiert, sondern vom Preis und der Anzahl der einzelnen Geräte.
