\subsection{Anforderungen}
An das Gerät und das gesamte System werden diverse Anforderungen gestellt, welche im Folgenden aufgezählt werden.

\subsubsection{Robustheit}
Das System muss eine sehr robuste Auswertung liefern, so dass es möglich ist, bei jeder Wettersituation dieselben Resultate zu erzielen. Im Allgemeinen bedeutet es, dass das System die Verkehrsteilnehmer eindeutig identifizieren können muss, egal ob die Verkehrsteilnehmer voller Schmutz, im Regen oder unter Sonneneinstrahlung sind. Das Gerät wird dazu in ein wasserdichtes Gehäuse gehüllt. 

\subsubsection{Laufzeit}
Das Gerät arbeitet eigenständig und ist mit einer externen Spannungsquelle versehen, da nicht an jeder Strassenlaterne ein Anschluss vorhanden ist, um die Geräte dort zu laden. Die Laufzeit der Geräte sollte mehrere Tage betragen, da nur so gewährleistet ist, dass aussagekräftige Ergebnisse im Bezug auf den Verkehrsfluss erziel werden können.  

\subsubsection{Verarbeitungszeit}
Die Verarbeitung der gesammelten Daten sollte beinahe in Echtzeit geschehen, um eventuell die Ergebnisse online darstellen zu können, wenn das System zu einem späteren Zeitpunkt weiter ausgereift wird. Aufgrund dessen wird ein sehr hoher Anspruch an die verwendete Hardware gesetzt. Ebenso muss die Software so angepasst sein, dass auch sie die Daten in Echtzeit verarbeiten kann.

\subsubsection{Kosten}
Man möchte jeder Gemeinde die Möglichkeit geben mit diesem System zu arbeiten. Damit es sich jede auch wirklich leisten kann, müssen die einzelnen Geräte sehr preisgünstig hergestellt werden. Die Vorgabe für ein einzelnes Gerät lautet deshalb, einen Preis von 200 CHF nicht zu überschreiten. Folglich ist es dann möglich für 1'000 CHF fünf solcher Geräte anzuschaffen. Das System wird dabei nicht durch die Qualität der einzelnen Bauteile definiert, sondern durch den Preis und das Zusammenspiel zwischen günstigen Bauteilen und gut durchdachter Software. Dadurch soll versucht werden mit den preisgünstigen Bauteilen eine gute Qualität des gesamten Gerätes zu erreichen.