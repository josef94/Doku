\subsection{Anforderungen}
An das Ger\"at und das gesamte System werden diverse Anforderungen gestellt welche im Folgenden aufgez\"ahlt werden.
\subsubsection{Robustheit}
Das System sollte eine sehr robuste Auswertung liefern, so dass es m\"oglich ist bei jeder Wettersituation die selben Resultate zu erziehlen. Im allgemeinen heisst dies dass das System die Verkehrsteilnehmer eindeutig identifizieren kann, egal ob die Verkehrsteilnehmer voller Schmutz, Regen oder Sonnenstrahlen sind. Das Ger\"at wird dazu in ein wasserdichtes Gehäuse gehüllt. 
\subsubsection{Laufzeit}
Das Ger\"at sollte autark arbeiten und mit einer externen Spannungsquelle versehen werde, da nicht an jeder Strassenlaterne ein Anschluss vorhanden ist um die Ger\"ate direkt zu Speisen. Die Laufzeit der autarken Ger\"ate sollte mehrere Tage betragen, da es sich um ein System handelt welches Daten braucht welche mehr als nur wenige Stunden sind. Um den Verkehrsfluss darstellen zu k\"onnen muss das Ger\"at mehrere Tage am St\"uck einsatzf\"ahig sein.
\subsubsection{Verarbeitungszeit}
Die Verarbeitung der gesammelten Daten sollte beinahe in Echtzeit geschehen, um eventuell bei sp\"ateren weiter Entwicklungen online dargestellt werden zu k\"onnen. Daraus wird ein sehr hoher Anspruch an die eingesetzte Hardware gesetzt. Ebenso muss die Software so angepasst sein, dass diese ebenso in Echtzeit die Daten verarbeitet hat.
\subsubsection{Kosten}
Um sich 