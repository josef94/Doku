\subsection{Video mittels Laptop}
Die ersten Videos für die Bachelorarbeit wurden mithilfe der Bildverarbeitungs-Bibliothek OpenCV auf einen Laptop aufgenommen. Zu diesem Zeitpunkt war die Aufgabe von OpenCV lediglich, die Videos Frame um Frame aufzunehmen und sie mit einem Videowriter aneinander zu reihen. Da der Prozessor des Laptops sehr leistungsstark war, konnten qualitativ ausgezeichnete Videos aufgenommen werden. Es ist lediglich aufgefallen, dass beim Abspielen des Videomaterials auf einem Computer die Geschwindigkeit erhöht war, wodurch ein 30-Minütiges Video in 15 Minuten vorüber war. Dies wurde zu diesem Zeitpunkt auf eine falsche Einstellung am Videowriter zurückgeführt. Da aber lediglich die einzelnen Frames relevant waren, konnte die Abspielgeschwindigkeit des Videos vernachlässigt werden. Durch das aufgenommene Videomaterial konnte schon in einer frühen Phase der Arbeit die Qualität der Kamera überprüft und Material für weitere Verarbeitungsschritte aufgenommen werden. Diese Aufnahmen entstanden an zwei Tagen in Ludesch und auf dem Parkplatz des NTBs. Obwohl die Kamera an beiden Orten nicht optimal positioniert wurde, konnten die Aufnahmen dennoch für viele Testprogramme und Szenarien verwendet werden. Zu diesem Zeitpunkt wurde die Option der Aufnahme via Laptop gewählt, da das Beaglebone noch nicht einsatzbereit war, jedoch parallel zur Hardware auch an der Software entwickelt werden sollte.