\subsection{Video mittels Beaglebone}
Nachdem alle notwendigen Programme installiert und zugehörigen Einstellungen auf dem Beaglebone abgeschlossen waren, konnten erste Videoaufnahmen mithilfe dieses Entwicklungsboards aufgenommen werden. Zu diesem Zeitpunkt war das Gerät bereits standalone und benötigte aus diesem Grund keine externe Stromzufuhr mehr. Der Testversuch wurde an einer Strassenlaterne in Buchs, auf einer Höhe von etwa fünf Metern durchgeführt. Die präzise Ausrichtung des Gerätes konnte durch das Mobilphone durchgeführt werden, da das Programm alle 10 Sekunden ein Bild abgespeichert und es auf einer Internetseite angezeigt hatte. Dort konnte zum ersten Mal Bilder mit optimaler Positionierung der Kamera aufgenommen werden. Die Aufgabe des Programms war es zu diesem Zeitpunkt, die Fahrzeuge aufzunehmen, mithilfe von OpenCV zu identifizieren und bei Identifikation eines Verkehrsteilnehmers zwei Bilder in einem Ordner abzuspeichern. Zudem sollte das komplette Video ebenfalls abgespeichert werden, damit dies für weitere Auswertungen verwendet werden konnte. Die Software selber funktionierte sehr gut. Leider musste jedoch festgestellt werden, dass durch die lange Prozedurdauer der Hauptschleife lediglich eine Bildrate von etwa einem FPS erzielt werden konnte. Da die Videos dennoch mit 30 FPS aneinandergereiht wurden, war ein 30-Minütiges Video in etwa zwei Minuten vorbei und somit konnte das Videomaterial praktisch gar nicht verwendet werden.\\
Diese Testaufnahmen zeigten, dass die Durchführung mittels eines Prozessors mit nur einem Kern praktisch unmöglich für diese Arbeit waren. Eine Möglichkeit, welche mit dem Beaglebone dennoch bestand, war es, nur Videos aufzunehmen und diese danach abzuspeichern. Da der Speicher des Beaglebones trotz Speicherkarte sehr schnell voll gewesen wäre, hätte die Speicherkarte des Geräts täglich gewechselt und die komplette Verarbeitung der Videos extern durchgeführt werden müssen. Da dies sehr umständlich gewesen wäre, musste alternativen gesucht werden. Aus diesem Grund folgte dann der Umstieg vom Beaglebone zum NanoPi, da dieser Prozessor mit vier Kernen ausgestattet war.