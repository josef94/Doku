\subsection{Video mittels Beaglebone}
Nachdem alle notwendigen Programme installiert und die zugehörigen Einstellungen auf dem Beaglebone abgeschlossen waren, konnten erste Videoaufnahmen mithilfe dieses Entwicklungsboards aufgenommen werden. Zu diesem Zeitpunkt war das Gerät bereits standalone und benötigte aus diesem Grund keine externe Stromzufuhr mehr. Der Testversuch wurde an einer Strassenlaterne in Buchs, auf einer Höhe von etwa fünf Metern, durchgeführt. Die präzise Ausrichtung des Gerätes konnte durch das Mobilphone erfolgen, da das Programm alle zehn Sekunden ein Bild abgespeichert und es auf einer Internetseite angezeigt hatte. Dort konnten dann zum ersten Mal Bilder mit optimaler Positionierung der Kamera aufgenommen werden. Die Aufgabe des Programms war es, die Fahrzeuge aufzunehmen, mithilfe von OpenCV zu identifizieren und bei erfolgreicher Identifikation eines Verkehrsteilnehmers zwei Bilder in einem Ordner abzuspeichern. Zudem sollte das komplette Video ebenfalls abgespeichert werden, damit dies für weitere Auswertungen verwendet werden konnte. Die Software selbst funktionierte sehr gut. Leider musste aber festgestellt werden, dass durch die lange Prozedurdauer der Hauptschleife nur eine Bildrate von etwa einem FPS erzielt werden konnte. Da die Videos dennoch mit 30 FPS aneinandergereiht wurden, war ein 30-minütiges Video in etwa zwei Minuten vorbei. Aus diesem Grund konnte das Videomaterial praktisch gar nicht verwendet werden.\\\\
Diese Testaufnahmen zeigten, dass die Durchführung mittels Prozessor mit nur einem Kern praktisch unmöglich für diese Arbeit war. Eine Möglichkeit, welche mit dem Beaglebone dennoch bestand, war es, nur Videos aufzunehmen und diese danach sofort abzuspeichern. Da der Speicher des Beaglebones trotz Speicherkarte sehr schnell voll gewesen wäre, hätte die Speicherkarte des Gerätes täglich gewechselt und die komplette Verarbeitung der Videos extern durchgeführt werden müssen. Da dies sehr umständlich gewesen wäre, mussten Alternativen gesucht werden. Aus diesem Grund folgte dann der Umstieg vom Beaglebone zum NanoPi, da dieser Prozessor mit vier Kernen ausgestattet war.