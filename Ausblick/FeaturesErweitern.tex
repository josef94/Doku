\subsection{Features erweitern}
Je besser der Feature Vektor ist, desto einfacher und genauer kann die anschliessende Verkehrsverfolgung durchgeführt werden. Aus diesem Grund wäre es vorteilhaft, wenn aus den vorhandenen Bildern mehr Features zu erkennen sind. Nachfolgend sind deshalb einige Möglichkeiten aufgelistet, wie der Feature Vektor um zusätzliche Spalten erweitert werden könnte.

\subsubsection{Fahrzeuglänge \& -breite}
Ein mögliches Feature, das realisiert werden könnte, wäre die Identifikation der Fahrzeuglänge und -breite. Im besten Falle könnte man dies direkt am vorhandenen Bild ableiten. Dazu müsste aber ein Referenzmass für beide Fahrbahnen in Sichtweite der Kamera aufgetragen werden. Das Herauslesen von Längenmassen ist jedoch auch über andere Sensoren realisierbar. 

\subsubsection{Kategorisieren}
Ebenfalls plausibel wäre beispielsweise das Kategorisieren der Verkehrsteilnehmer in mehrere Unterklassen. Mögliche Unterteilungen sind hierbei unter anderem Fahrrad, Motorrad, PKW, Lieferwagen, Lastwagen oder Traktor. Die Unterscheidung könnte man anhand des Verhältnisses zwischen Länge und Breite und anhand der Pixelanzahl der Blobs erhalten. Die im Abschnitt Testversuche geschilderte Kategorisierung könnte dabei auf mehrere Klassen ausgeweitet werden.


\subsubsection{Farbe}
Die Farbe des Fahrzeuges ist ein weiteres wichtiges Merkmal, womit es erneut identifiziert werden kann. Damit dieses Feature gegenüber verschiedenen Belichtungen dennoch genügend resistent wäre, müsste dabei die erkannte Farbe in eine von etwa zehn Möglichkeiten eingeteilt werden. Die Tageszeit spielt dabei eine grosse Rolle, da eine Farberkennung unmöglich ist, wenn es draussen dunkel ist. Die Unterteilung in eine der zehn Farben wäre mithilfe des "'K-Means Clustering"'-Algorithmus plausibel.

\subsubsection{Farbfläche}
Je grösser ein Auto ist, desto mehr Farbflächen sind vorhanden. Die Anzahl der Pixel dieser Farbflächen könnte, neben der Farbe an sich, ebenfalls dem Feature Vektor ergänzt werden. Damit eine Unterscheidung zwischen Fensterscheibe und Farbfläche stattfinden kann, müsste vorgängig ein Clustering auf eine einheitliche Farbe durchgeführt werden.

\subsubsection{Sonderanbringungen}
Auf einigen Fahrzeugen befinden sich Sonderanbringungen wie Beschriftungen oder Bilder. Wenn diese Sonderanbringungen erkannt und bestenfalls abgespeichert werden könnten, so wäre dadurch eine eindeutige Wiedererkennung dieses Verkehrsteilnehmers denkbar. Realisierbar wäre dies, wenn ein Algorithmus eine Sonderanbringung erkennt und davon ein zusätzliches Bild abspeichert.

\subsubsection{Nummerntafeln}
Mithilfe der Nummerntafeln sind die meisten Verkehrsteilnehmer ohne grossen Aufwand verfolgbar, da es nur wenige Verkehrsteilnehmer gibt, die keine Nummerntafel besitzen. Es könnte jedoch zum Problem werden, da die Ausrichtung der Kamera an der falschen Position ist und deshalb das Kennzeichen nicht richtig erkannt wird. Ebenso darf eine Nummerntafel aus rechtlichen Gründen eigentlich nicht ohne Weiteres aufgenommen werden, da damit ein Verkehrsteilnehmer eindeutig identifiziert werden kann.

\subsubsection{Geschwindigkeit}
Für eine Verkehrsverfolgung ist die Geschwindigkeit kaum relevant, da sie sich während der Fahrt kontinuierlich ändern kann. Jedoch wäre diese Information für die Gemeinden selbst nützlich. Mit der Geschwindigkeit der Verkehrsteilnehmer könnte ein Durchschnitt ermittelt werden, der als Indikator dient, wie schnell tatsächlich am Gerät vorbeigefahren wurde. Die geschilderte Geschwindigkeitserkennung im Kapitel Testversuche wäre dabei eine Variante, die verwendet werden könnte. Um jedoch genauere Daten zu erhalten, müssten bestenfalls andere Sensoren verwendet werden.