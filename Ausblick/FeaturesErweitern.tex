\subsection{Features erweitern}
Je besser der Feature Vektor ist, desto einfacher und genauer kann die im Anschluss folgende Verkehrsverfolgung durchgeführt werden. Aus diesem Grund wäre es notwendig, aus den vorhandenen Bildern mehr Features zu erkennen. Nachfolgend sind deshalb einige Möglichkeiten aufgelistet, wie der Feature Vektor um zusätzliche Spalten erweitert werden könnte.

\subsubsection{Kategorisieren}
Ein mögliches Feature, das realisiert werden könnte wäre beispielsweise das Kategorisieren der Verkehrsteilnehmer in mehrere Unterklassen. Mögliche Unterteilungen wären hierbei Fahrrad, Motorrad, PKW, Lieferwagen, Lastwagen, Traktor usw. Realisiert werden könnte dies über das Verhältnis der Länge und Breite und gleichzeitig der Anzahl an Pixeln der Blobs.

\subsubsection{Fahrzeuglänge \& -breite}
Ebenfalls plausibel könnte die Identifikation der Fahrzeuglänge oder -breite sein. Im besten Fall wäre diese direkt über das vorhandene Bild realisierbar. Dazu müsste aber ein Referenzmass für beide Fahrbahnen in Sichtweite der Kamera aufgetragen werden. Das herauslesen von Längenmassen könnte jedoch auch über andere Sensoren realisiert werden.

\subsubsection{Farbe}
Die Farbe des Fahrzeugs wäre ein mögliches Merkmal, das unterschieden werden könnte. Damit dieses Feature gegenüber verschiedenen Belichtungen dennoch genügend resistent wäre, müsste dabei die erkannte Farbe in eine von etwa zehn Möglichkeiten eingeteilt werden. Ebenso müsste hier die Tageszeit beachtet werden, da eine Farbeerkennung unmöglich ist, wenn es draussen dunkel ist.

\subsubsection{Farbfläche}
Je grösser ein Auto ist, desto mehr Farbflächen sind vorhanden. Die Anzahl Pixel dieser Farbflächen könnte neben der Farbe ebenfalls dem Feature Vektor ergänzt werden. Damit eine Unterscheidung zwischen Fensterscheibe und Farbfläche durchgeführt werden kann, müsste vorgängig ein Clustering auf eine einheitliche Farbe durchgeführt werden.

\subsubsection{Sonderanbringungen}
Auf einigen Fahrzeugen befinden sich Sonderanbringungen wie Beschriftungen oder Bilder. Wenn diese Sonderanbringungen erkannt und bestenfalls abgespeichert werden könnten, so wäre dadurch eine eindeutige Wiedererkennung dieses Verkehrsteilnehmers realisierbar. Durchführbar wäre dies, wenn ein Algorithmus eine Sonderanbringung erkennt und davon ein zusätzliches Bild abspeichert.

\subsubsection{Nummerntafeln}
Mithilfe der Nummerntafeln wären die meisten Verkehrsteilnehmer ohne grossen Aufwand verfolgbar, da es nur wenig Fahrzeuge gibt die keine Nummerntafel besitzen. Problematisch wäre jedoch, dass die Ausrichtung der Kamera zum Erkennen der Kennzeichen an der falschen Position ist. Ebenso darf eine Nummerntafel aus rechtlichen Gründen nicht ohne weiteres aufgenommen werden, da damit ein Verkehrsteilnehmer eindeutig identifiziert werden kann.

\subsubsection{Geschwindigkeit}
Obwohl die Geschwindigkeit kaum für eine Verkehrsverfolgung genutzt werden kann, da sich diese während der Fahrt kontinuierlich ändern kann, so wäre diese Information für Gemeinden nützlich. Mit der Geschwindigkeit der Verkehrsteilnehmer könnte ein Durchschnitt ermittelt werden, der als Indikator dient, wie schnell tatsächlich am Gerät vorbeigefahren wurde.