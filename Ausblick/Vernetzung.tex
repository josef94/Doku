\subsection{Vernetzung}
Im Moment funktioniert die Auswertung, indem die Daten vor Ort abgeholt und danach auf einem Rechner verarbeitet werden. Es wäre denkbar, dass die Geräte über ein Lora-WAN Netzwerk miteinander verbunden werden. Dadurch könnten die vorhandenen Feature Vektoren direkt übertragen werden, weshalb kein manuelles Abholen der Daten mehr notwendig ist. Das Monitoring der Geräte wäre so komplett online möglich. Zudem ist eine Überprüfung der Temperaturen und Ausrichtung der Kamera jederzeit möglich. Lediglich eine Feinjustierung der Kamera und das Auswechseln des Akkus müsste weiterhin vor Ort stattfinden.