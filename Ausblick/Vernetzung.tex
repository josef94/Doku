\subsection{Vernetzung}
Im Moment funktioniert die Auswertung, indem die Daten vor Ort abgeholt und danach auf einem Rechner verarbeitet werden. Es wäre denkbar, dass die Geräte über ein Lora-WAN Netzwerk miteinander verbunden wären. Somit könnten die vorhandenen Feature Vektoren direkt übertragen werden, weshalb kein manuelles Abholen der Daten mehr nötig wäre. Das Monitoring der Geräte wäre so komplett online möglich. Die Temperaturen und Ausrichtung der Kamera könnten jederzeit überprüft werden. Lediglich eine Feinjustierung der Kamera und das Auswechseln des Akkus müsste vor Ort durchgeführt werden.