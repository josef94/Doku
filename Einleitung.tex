\section{Einleitung}
\subsection{Aufgabenstellung}
Die Aufgabe des Projektes war es, ein Gerät zu entwickeln, welches autonom vorbeifahrende Verkehrsteilnehmer erkennt und im Anschluss diverse Operationen auf diese ausführt, um somit eine genaue Identifikation dieser Verkehrsteilnehmer zu erhalten. Das Gerät sollte zwei verschiedene Zähler beinhalten, welche sowohl links- als auch rechtsfahrende Fahrzeuge unabhängig voneinander zählen kann. Zudem sollte das Gerät die aufgenommenen Fahrzeuge je nach Grösse in verschiedene Kategorien unterteilen und ebenfalls eine grobe Abschätzung über die momentane Geschwindigkeit treffen können. Ein zusätzlicher Zeitstempel, welcher bei der Vorbeifahrt der einzelnen Fahrzeuge aufgenommen wurde, trägt alle gefunden Indikatoren in eine Tabelle ein, damit diese für spätere Auswertungen verwendet werden können. Eine wichtige Voraussetzung war es, dass ein Gerät etwa eine Woche ohne Unterbrechung im Betrieb bleiben kann und zudem die eingesetzte Speicherkarte genügend Kapazität aufweist, damit diese im besagten Zeitraum nicht überfüllt wird. Ausserdem wurde eine möglichst einfache Montage der einzelnen Geräte angestrebt, wobei dennoch ein gewisses Monitoring möglich ist. Im besten Fall soll die Ausrichtung mit einem Mobilphone durchgeführt werden können und somit die momentane Ausrichtung ohne grossen Zeitaufwand kontrolliert und korrigiert werden können.\\\\
Um eine aussagekräftige Verkehrsverfolgung zu erhalten, werden die einzelnen Geräte zu einem kompletten System gekoppelt. So sollen die Fahrzeuge an mehreren Orten von den verschiedenen Geräten wiedererkannt und damit der Bewegungsablauf der einzelnen Verkehrsteilnehmer rekonstruiert werden. Durch strategisch günstige Positionen der Geräte und geeignete Algorithmen soll es möglich sein, dass nur ein Bruchteil aller aufgenommenen Fahrzeuge für die Identifikation eines bestimmten Teilnehmers in Betracht kommen kann. Dadurch wird eine genauere und schnellere Auswertung erreicht. Es muss nicht zwingend jeder Verkehrsteilnehmer erkannt werden, da es sich dabei lediglich um eine statistische Aussage handelt. Diese Entscheidung wurde so festgelegt, da nicht alle Verkehrsteilnehmer zu 100\% verfolgt werden können. Probleme entstehen beispielsweise, wenn diese im Ort wohnen oder für längere Zeit dort verweilen.\\\\
Damit eine Verkehrsverfolgung gelingen kann, müssen neben den bereits erwähnten Indikatoren noch weitere Eigenschaften gefunden werden, um die Fahrzeuge genauer einteilen zu können. Welche zusätzlichen Indikatoren damit gemeint sind, sollte sich im Verlauf dieser Bachelorarbeit zeigen. Letztendlich soll die Auswertung der aufgenommenen Resultate noch visualisiert werden. Bestenfalls könnte dies mithilfe von Open Source Tools wie beispielsweise Open Street Map geschehen.
Die Aufgabe dieses Projektes war es, ein Gerät zu entwickeln, welches autonom vorbeifahrende Verkehrsteilnehmer erkennt und im Anschluss diverse Operationen auf diese ausführt, um somit eine genaue Identifikation dieser Verkehrsteilnehmer zu erreichen. Das Gerät sollte zwei verschiedene Zähler beinhalten, welche sowohl links als auch rechtsfahrende Fahrzeuge unabhängig voneinander zählen kann. Zudem sollte das Gerät die aufgenommenen Fahrzeuge je nach Grösse in verschiedene Kategorien unterteilen und ebenfalls eine grobe Abschätzung über die momentane Geschwindigkeit aussagen können.  Mit zusätzlichem Zeitstempel, welcher bei der Vorbeifahrt der einzelnen Fahrzeuge aufgenommen wurde, sollten alle gefunden Indikatoren in eine Tabelle eingetragen werden, damit diese für spätere Auswertungen verwendet werden kann.  Ein Gerät sollte etwa eine Woche ohne Unterbruch im Betrieb bleiben und zusätzlich die Speicherkarte in dieser besagten Zeit nicht überfüllen. Das einzelne Gerät soll möglichst einfach aufgestellt werden können und dennoch ein gewisses Monitoring möglich sein. Im besten Fall soll eine Ausrichtung mithilfe des Mobilphones durchgeführt werden können und ebenfalls die momentane Ausrichtung ohne grossen Zeitaufwand kontrolliert und korrigiert werden können.\\
Damit eine Verkehrsverfolgung durchgeführt werden kann, sollen die einzelnen Geräte zu einem System gekoppelt werden können. So sollen die Fahrzeuge an mehreren Orten von mehreren Geräten wiedererkannt werden und damit der stattgefundene Verkehrsfluss der Teilnehmer rekonstruiert werden können. Durch strategisch günstige Positionen der Geräte und geeignete Algorithmen sollen nur ein Bruchteil aller aufgenommenen Fahrzeuge für die Wiederidentifikation eines Teilnehmers in Betracht kommen und somit eine genauere und schnellere und Auswertung erreicht werden. Dabei soll es sich um eine statische Aussage handeln, weshalb nicht zwingend jeder Verkehrsteilnehmer exakt erkannt werden muss. Dies wurde so bestimmt, da die Verkehrsteilnehmer "'verschwinden"' können, falls diese im Ort wohnen oder für längere Zeit anderweitig beschäftigt sind.\\
Damit die Verkehrsverfolgung gelingen kann, müssen neben den bereits erwähnten Indikatoren noch weitere Eigenschaften gefunden werden, um die Fahrzeuge genauer einteilen zu können. Was für Indikatoren der Tabelle hinzugefügt werden, sollte sich im Verlauf dieser Bachelorarbeit zeigen. Eine gewisse Auswertung der aufgenommenen Resultate solle ebenfalls visualisiert werden. Bestenfalls könnte dies mithilfe von Open Source Tools wie beispielsweise Open Street Map geschehen.

\subsection{Wertschöpfung}
Das System soll Verkehrsplaneren dazu dienen, ihre Planungen durchführen zu können und so die Fahrgewohnheiten von einzelnen Teilnehmern genauer verstehen und nachvollziehen zu können. Durch geschicktes Aufstellen von Hindernissen oder Geschwindigkeitsbegrenzungen soll die Veränderung des Flusses nachvollzogen werden und somit durch optimale Aufstellung der Verkehrssignale verhindert werden, den Verkehr durch ungewollte Gebiete – wie Wohnsiedlungen – zu lenken. Da Verkehrszählungen zurzeit manuell und aufgrund der Konzentration der Personen nur für eine kurze, begrenzte Zeit durchgeführt werden, soll dies mithilfe dieses Gerätes automatisch und bestenfalls über längere Zeiten durchgeführt werden können.\\
Ebenso soll dieses System von Geräten von Gemeinden genutzt werden können, um deren Geschwindigkeiten, Anzahl Teilnehmer oder Art des Fahrzeugs an den Standorten analysieren zu können und somit gezielt an Orten diverse Veränderungen durchführen zu können. 
Das System soll Verkehrsplaneren dazu dienen, ihre Planungen durchführen zu können und so die Fahrgewohnheiten von einzelnen Teilnehmern genauer verstehen und nachvollziehen zu können. Durch geschicktes Aufstellen von Hindernissen oder Geschwindigkeitsbegrenzungen soll die Veränderung des Flusses nachvollzogen werden und somit durch optimale Aufstellung der Verkehrssignale verhindert werden, den Verkehr durch ungewollte Gebiete – wie Wohnsiedlungen – zu lenken. Da Verkehrszählungen zurzeit manuell und aufgrund der Konzentration der Personen nur für eine kurze, begrenzte Zeit durchgeführt werden, soll dies mithilfe dieses Gerätes automatisch und bestenfalls über längere Zeiten durchgeführt werden können.\\
Ebenso soll dieses System von Geräten von Gemeinden genutzt werden können, um deren Geschwindigkeiten, Anzahl Teilnehmer oder Art des Fahrzeugs an den Standorten analysieren zu können und somit gezielt an Orten diverse Veränderungen durchführen zu können. 
\newpage