\section{Abstract}
Das Ziel von Fast and Curious war es, ein System zu entwickeln, welches als einzelnes Gerät vorbeifahrende Verkehrsteilnehmer zählt, kategorisiert und eine grobe Aussage über dessen Geschwindigkeit aussagen kann. Als System von Geräten sollte es zudem eine statistische Aussage über das Fahrverhalten der aufgezeichneten Verkehrsteilnehmer treffen können. Ebenfalls mussten die Rahmenbedingungen von tiefen Kosten und einer langen Laufzeit eingehalten werden. Das gesamte System sollte bei jeder Witterung einsatzfähig sein und am Strassenrand einer maximal zweispurigen Strasse platziert werden. Das Gerät sollte an einer Strassenlaterne auf einer Höhe von sechs Meter befestigt werden.\\\\
Ein einzelnes Gerät besteht aus einer Rechnereinheit mit einem Quad-core Prozessor, welcher in der Lage ist, mehrere Prozesse parallel abzuarbeiten. Dies ist notwendig um eine beinahe Echtzeitauswertung gewährleisten zu können. Um die Verkehrsteilnehmer einzeln zu identifizieren, wird eine Kamera mit hoher Bildrate, jedoch geringer Auflösung verwendet. Um eine Messung an jeder beliebigen Strassenlaterne durchführen zu können, wird eine externe Stromversorgung benötigt, welche das Gerät über mehrere Tage hinweg einsatzfähig macht. Damit sich die Geräte in der richtigen Position befinden, wird mithilfe eines WIFI-Adapters eine Verbindung zu einem Endgerät herstellt. Dabei muss vom Benutzer ein Hotspot errichtet werden, auf welchen sich das Gerät verbindet. Mithilfe einer Website kann im Anschluss das Gerät gesteuert und überwacht werden. Um das Gerät parallel zur Strasse ausrichten zu können, wird auf der Website ein Live-Stream der Kameraposition angezeigt. Dadurch kann die genaue Ausrichtung justiert und das Gerät im Anschluss gestartet werden.\\\\
Im Betriebsmodus beschäftigt sich der erste Kern permanent mit der Aufnahme von Videos, welche nach 15 Minuten abgespeichert werden. Ein weiterer Kern analysiert die aufgenommenen Videos, indem jeweils zwei aufeinanderfolgende Frames verglichen werden. Somit können Bewegungen erkannt und Bilder mit Bewegungen temporär gespeichert werden. Nachdem das Video abgearbeitet wurde, wird es vom temporären Speicher entfernt. Der dritte Kern verwendet die Bilder, auf welchen Bewegungen erkannt wurden und verknüpft diese logisch miteinander um einen Verkehrsteilnehmer identifizieren zu können. Daraus werden zu jedem Verkehrsteilnehmer Eigenschaften extrahiert und in einem Feature Vektor gespeichert. Zu den extrahierten Eigenschaften gehört ein Zeitstempel, die Fahrtrichtung, ein Zähler für die Anzahl Fahrzeuge pro Fahrtrichtung, eine Kategorisierung sowie eine Geschwindigkeitseinteilung. Dieser Feature Vektor wird später verwendet, um den Verkehrsfluss in einem begrenzten Gebiet rekonstruieren und daraus eine statistische Aussage treffen zu können. Der letzte Kern beschäftigt sich mit der Steuerung und Überwachung der anderen Kerne. Dies ist notwendig, um garantieren zu können, dass den Prozessen jederzeit genügend Ressourcen zur Verfügung stehen. Und ausserdem, um kleinere Aufgaben nebenbei durchführen zu können. Dies beinhaltet das Monitoring der Rechnereinheit sowie das Handeln des Webservers, damit auf Eingaben des Benutzers reagiert werden kann.\\\\
Nachdem das Gerät entwickelt und ersten Tests standgehalten hatte, wurde ein erster Feldversuch mit mehreren Geräten in der Gemeinde Satteins im Vorarlberg durchgeführt. Dabei wurden sechs Geräte zur selben Zeit an verkehrsrelevanten Stellen angebracht und über eine Dauer von mehreren Tagen in Betrieb gehalten. Aus den gewonnenen Daten wurde zum einen die Anzahl Verkehrsteilnehmer an jedem Gerät gezählt und kategorisiert, zum anderen wurde der Verkehrsfluss in der Gemeinde Satteins statistisch rekonstruiert und dargestellt. Anhand von Testbildern, welche gleichzeitig aufgenommen wurden, konnte eine hohe Übereinstimmung des Verkehrsflusses nachgestellt werden.\\\\
In weiteren Arbeiten könnten die Geräte eine Echtzeitauswertung realisieren, welche jederzeit online abrufbar wäre. Dazu müssten die Geräte entweder ihre Daten auf einen Online-Server transferieren, oder mittels LoRa-WAN an einen Host übertragen. Ebenfalls wäre es möglich, die Geräte mit zusätzlichen Sensoren zu erweitern, sodass weitere verkehrsrelevante Daten wie Lautstärkemessungen, Wettersituation oder Schadstoffmessungen aufgenommen werden.
\newpage