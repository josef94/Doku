\section{Abstract}
Das Ziel von "'Fast and Curious"' war es, ein System zu entwickeln, welches aus einzelnen Geräten besteht, die vorbeifahrende Verkehrsteilnehmer zählen, kategorisieren und deren grobe Geschwindigkeit einschätzen können. Durch den Zusammenschluss der Geräte, sollte es möglich sein eine statistische Aussage über das gesamte Fahrverhalten der aufgezeichneten Verkehrsteilnehmer treffen zu können. Dabei mussten bestimmte Rahmenbedingungen, wie niedrige Kosten und lange Laufzeit eingehalten werden. Das gesamte System sollte bei jeder Witterung einsatzfähig sein und am Strassenrand einer maximal zweispurigen Strasse platziert werden. Die einzelnen Geräte werden an Strassenlaternen auf einer Höhe von sechs Meter befestigt.\\\\
Ein einzelnes Gerät besteht aus einer Rechnereinheit mit einem Quad-core Prozessor, welcher in der Lage ist, mehrere Prozesse parallel durchzuführen. Dies ist notwendig um eine beinahe Echtzeitauswertung gewährleisten zu können. Um die Verkehrsteilnehmer einzeln zu identifizieren, wird eine Kamera mit hoher Bildrate, jedoch geringer Auflösung, verwendet. Die Geräte werden mit einer externen Stromversorgung ausgestatten, welche die Einsatzfähigkeit über mehrere Tage hinweg gewährleistet. Ebenso ist es dadurch möglich sie an jeder beliebigen Strassenlaterne anzubringen und Messungen durchführen zu können. Damit sich die Geräte in der richtigen Position befinden, wird mithilfe eines WIFI-Adapters eine Verbindung zu einem Endgerät herstellt. Dabei muss vom Benutzer ein Hotspot errichtet werden, auf welchen sich das Gerät verbindet. Mithilfe einer Website kann im Anschluss das Gerät gesteuert und überwacht werden. Um das Gerät parallel zur Strasse ausrichten zu können, wird auf der Website ein Live-Stream der Kameraposition angezeigt. Dadurch kann die genaue Ausrichtung justiert und das Gerät im Anschluss gestartet werden.\\\\
Im Betriebsmodus beschäftigt sich der erste Kern permanent mit der Aufnahme von Videos, welche nach 15 Minuten abgespeichert werden. Ein weiterer Kern analysiert die aufgenommenen Videos, indem jeweils zwei aufeinanderfolgende Frames verglichen werden. Somit können Bewegungen erkannt und Bilder mit Bewegungen temporär gespeichert werden. Nachdem das Video abgearbeitet wurde, wird es vom temporären Speicher entfernt. Der dritte Kern verwendet die Bilder, auf welchen Bewegungen erkannt wurden und verknüpft diese logisch miteinander um einen Verkehrsteilnehmer identifizieren zu können. Daraus werden zu jedem Verkehrsteilnehmer Eigenschaften extrahiert und in einem Feature Vektor gespeichert. Zu den extrahierten Eigenschaften gehört unter anderem ein Zeitstempel, die Fahrtrichtung, ein Zähler für die Anzahl der Fahrzeuge pro Fahrtrichtung, eine Kategorisierung, sowie eine Geschwindigkeitseinteilung. Dieser Feature Vektor wird später verwendet, um den Verkehrsfluss in einem begrenzten Gebiet rekonstruieren und daraus eine statistische Aussage treffen zu können. Der letzte Kern beschäftigt sich mit der Steuerung und Überwachung der anderen Kerne. Dies ist notwendig, um garantieren zu können, dass den Prozessen jederzeit genügend Ressourcen zur Verfügung stehen. Ausserdem ermöglicht ein kleiner zeitlicher Puffer, dass nebenbei noch kleinere Aufgaben durchgeführt werden können. Dies sind beispielsweise das Monitoring der Rechnereinheit sowie das Handeln des Webservers, damit unverzüglich auf Eingaben des Benutzers reagiert werden kann.\\\\
Nachdem das Gerät entwickelt und den ersten Tests standgehalten hatte, wurde ein erster Feldversuch mit mehreren Geräten in der Gemeinde Satteins im Vorarlberg durchgeführt. Dabei wurden sechs Geräte zur selben Zeit an verkehrsrelevanten Stellen angebracht und über eine Dauer von mehreren Tagen in Betrieb genommen. Aus den gewonnenen Daten geht hervor, dass zum einen die Anzahl Verkehrsteilnehmer an jedem Gerät gezählt und kategorisiert, zum anderen der Verkehrsfluss in der Gemeinde Satteins statistisch rekonstruiert und dargestellt wurde. Anhand von Testbildern, welche parallel aufgenommen wurden, konnte der nachgestellte Verkehrsfluss überprüft werden.\\\\

\newpage
Mit den Geräten wäre es möglich eine Echtzeitauswertung zu realisieren, welche jederzeit online abgerufen werden kann. Dazu müssten diese ihre Daten aber entweder auf einen Online-Server transferieren oder mittels LoRa-WAN an einen Host übertragen. LoRa-WAN ist ein Long-Range Wirelessprotokoll, welches über mehrere Kilometer hinweg Daten versenden kann. Die zu übertragenden Daten dürfen dabei nicht zu gross sein, da LoRa-WAN nur eine geringe Übertragungsgeschwindigkeit bietet. Ebenfalls besteht die Möglichkeit, die Geräte mit zusätzlichen Sensoren zu erweitern, welche weitere verkehrsrelevante Daten wie Lautstärkemessungen, Wettersituation oder Schadstoffmessungen aufnehmen könnten.
\newpage