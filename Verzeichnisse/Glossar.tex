\subsection{Glossar}
% C:\Users\Dave\Documents\Sourcetree\Doku> makeglossaries Dokumentation

\newglossaryentry{Frame}
{
  name=Frame,
  description={Ein Frame bezeichnet eine Einzelbild, welches aus einem Video extrahiert wurde. \cite{frame}}
}

\newglossaryentry{Feature Vektor}
{
  name=Feature Vektor,
  description={Ein Feature Vektor umfasst die parametrisierbaren Eigenschaften eines Musters in vektorieller Weise. Verschiedene, für das Muster charakteristische Merkmale bilden die verschiedenen Einträge dieses Vektors. \cite{featureVector}}
}

\newglossaryentry{Monitoring}
{
  name=Monitoring,
  description={Als Monitoring bezeichnet man das Messen, Überwachen und Beobachten eines Prozesses. \cite{monitoring}}
}

\newglossaryentry{Open Source Tool}
{
  name=Open Source Tool,
  description={Open Source Tool definiert öffentlichen Quelltext von Software, welche kostenlos eingesehen, genutzt oder geändert werden kann. \cite{ost}}
}

\newglossaryentry{Open Street Map}
{
  name=Open Street Map,
  description={Open Street Map ist ein freies Projekt, welches dazu dient, frei nutzbare Geodaten für jeden frei zu Verfügung zu stellen. \cite{osm}}
}

\newglossaryentry{Framerate}
{
  name=Framerate,
  description={Die Framerate bezeichnet die Anzahl der Einzelbilder pro Zeiteinheit, welche aufgenommen und wiedergegeben werden. \cite{framerate}}
}

\newglossaryentry{Plugin}
{
  name=Plugin,
  description={Es handelt sich dabei um optionale Software-Module, welche die Grundfunktionen der Software erweitern. \cite{plugin}}
}

\newglossaryentry{Blob detection}
{
  name=Blob detection,
  description={In der Bildverarbeitung benutze Methode, um ein Objekt mit konstanten Eigenschaften zu detektieren. \cite{blobDet}}
}

\newglossaryentry{Blob tracking}
{
  name=Blob tracking,
  description={Benutzte Methode in der Bildverarbeitung, um ein detektiertes Objekt über mehrere Bilder hinweg zu verfolgen}
}


\newglossaryentry{Git}
{
  name=Git,
  description={Freie Software zur verteilten Versionsverwaltung von Dateien. \cite{git}}
}

\newglossaryentry{SourceTree}
{
  name=SourceTree,
  description={SourceTree ist ein Programm für Windows, mit welchem es ermöglicht wird, einfachen Zugriff auf ein Git-Repository herzustellen. \cite{sourceTree}}
}

\newglossaryentry{Repository}
{
  name=Repository,
  description={Ein Repository ist ein verwaltetes Verzeichnis zur Speicherung und Beschreibung von digitalen Objekten.  \cite{rep}}
}

\newglossaryentry{SoftAP-Mode}
{
  name=SoftAP-Mode,
  description={Software aktivierter Verbindungspunkt, ohne dabei in einem Netzwerk eingebunden zu sein}
}

\newglossaryentry{PuTTY}
{
  name=PuTTY,
  description={PuTTY ist eine freie Software die verwendet wird, um über verschiedene Schnittstellen auf andere Geräte und Objekte zuzugreifen.  \cite{putt}}
}

\newglossaryentry{Cluster}
{
  name=Cluster,
  description={Ein Cluster definiert eine Gruppe von Objekten, welche ähnliche Eigenschaften aufweisen.  \cite{cluster}}
}

\newglossaryentry{MySQL}
{
  name=MySQL,
  description={MySQL ist ein Open Source Tool für relationalen Datenbanken. Es ist für mehrere Betriebssysteme verfügbar und beinhaltet zudem eine Online-Schnittstelle. \cite{mySQL}}
}

\newglossaryentry{Eduroam}
{
  name=Eduroam,
  description={Eduroam steht für "'education roaming"'. Mithilfe von Eduroam kann mit einem Benutzer einer teilnehmenden Organisation auf jedes WLAN-Netzwerk anderer teilnehmender Organisationen beigetreten werden. \cite{eduroam}}
}
\glsaddall
\printglossary[style=super, nonumberlist]